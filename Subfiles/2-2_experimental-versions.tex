
\subsubsection{Thermal Grooving}

The thermal grooving technique was examined to provide a comparison point to our proposed gallium contact angle method.  These thermal grooves appear at grain boundaries, but the most information can be drawn from grain boundaries formed by two different crystal orientations. Using electron backscatter diffraction, EBSD, we identified grain orientations on samples of galfenol and alfenol to identify the grain boundaries where \hkl(100), \hkl(110), and \hkl(111) orientations met.  By measuring the dihedral angle formed at these grain boundaries after polishing and annealing, we calculated the ratio of grain boundary energy and surface energy according to the following equation: 

\begin{figure}[h]
	\centering
	\begin{subfigure}[c]{0.45\textwidth}
		\includegraphics[width=\linewidth]{afm-groove-fega}
		\subcaption{~}
		\label{fig:afm-groove-fega}		
	\end{subfigure}
	\begin{subfigure}[c]{0.45\textwidth} 
		\includegraphics[width=\linewidth]{fega-groove-profile}
		\subcaption{~}
		\label{fig:fega-groove-profile}		
	\end{subfigure}
	\caption{(a) 3D rendering of (110)/(111) grain boundary on surface of  for (Fe-19\%Ga)+1.0\%NbC sample where the depth of groove is ~8 nm. (b) A quadratic fit of a (110)/(111) grain boundary profile for a (Fe-19\%Ga)+1.0\%NbC sample.}
	\label{fig:thermal-groove}
\end{figure}
\begin{equation}
	\frac{\gamma_{GB}}{\gamma_{S}} = 2\cos\left(\frac{\Psi_{S}}{2}\right) 
\end{equation}
where $\gamma_{GB}$ is the grain boundary energy, $\gamma_{S}  $ is the surface energy, and $\Psi_{S} $ is the dihedral angle, as described in Rohrer et al.\cite{Rohrer2010a} The most symmetric thermal groove came from a \hkl(110)/\hkl(111) grain boundary on a rolled and annealed (Fe-19\%Ga)+1.0\%NbC sample made by Suok-Min Na, as seen in the Figure \ref{fig:thermal-groove}.  The grain boundary profiles were measured using atomic force microscopy (AFM), and the dihedral angles were extrapolated from the profiles using both AFM software by Bruker and, the more accurate, quadratic fit.  The results are shown in Table \ref{groove-analysis}.  This groove in particular had a depth of $\sim$8 nm which is significantly smaller in depth compared to grooves of other metal alloys in literature.  Also, many of the grain boundary grooves were not suitable for measuring based on their lack of symmetry at the grain boundary interface. It is worthy to note that this technique is very time consuming and slightly destructive to the surface. To properly analyze the thermal grooving technique, an extensive grain boundary study of annealing temperatures and times on Alfenol and Galfenol would have to be carried out. While this may be an interesting avenue of research in the future, the resultant calculations of relative grain energies does not benefit our ultimate goal of achieving a comprehensive AGG model for Galfenol and Alfenol. Realization of our goal lies in the measurement of orientation-dependent surface energy using contact angle measurements. 


\begin{table}[h!]
	\centering
	\caption{Calculated dihedral angles and relative energies from our most symmetric grain boundary groove.}
	\begin{tabular} { |p{1cm}||c|c|c|c|  } 
		\hline
		\multicolumn{5}{|c|}{fe-ga-s12-006 profile analysis - GB (110)/(111)}\\
		\hline
		~	&\multicolumn{2}{|c|}{Bruker Software}		&\multicolumn{2}{|c|}{Quadratic Fit}	\\
		\hline
		Profile	&Dihedral Angle (\degree)	&Relative Energy	&Dihedral Angle (\degree)	&Relative Energy \\ 
		\hline
		1		&156.957	&0.399471	&151.253	&0.496475	\\
		\hline
		2		&156.244	&0.411657	&157.093	&0.397144	\\
		\hline
		3		&154.402	&0.443063	&155.554	&0.423432	\\
		\hline
		4		&157.221	&0.394955	&152.785	&0.470541	\\
		\hline
		5		&154.732	&0.437445	&154.966	&0.433458	\\
		\hline
		6		&158.386	&0.375003	&154.482	&0.441706	\\
		\hline
		\textbf{Avg}	&156.324$\pm$1.52919	&0.410266$\pm$0.0261178	&154.356$\pm$2.06926	&0.443793$\pm$0.0351932\\
		\hline
	\end{tabular}
	\label{groove-analysis}
\end{table}

	


\subsubsection{Contact Angle Goniometer (Verison 1)}
Preliminary designs of our contact angle goniometer implement a radiative temperature control box which encloses an argon gas filled container where the sample resides, as seen in Figure \ref{fig:rad-temp-box}.  The presence of argon is meant to prevent any further oxidation of the Galfenol sample as well as the gallium droplet. The argon filled container was initially made of a clear acrylic plastic, but prolonged exposure to temperatures above 80\degree C caused thermal deformation of the plastic making longer experiments impossible to perform without environment contamination.  A clear pyrex container replaced the acrylic box to fix this issue.  Application of the liquid gallium drop to our surfaces was done via a mounted plastic syringe with commercially available disposable stainless steel hypodermic needles.  Observations of this application show that the hypodermic needles present multiple problems to the sessile drop method.  Liquid gallium tends to adhere strongly to the stainless steel needle tips which makes wetting to the sample very difficult.  Tapping the syringe will remove the drop needle and allow gallium to wet the surface, but this is not ideal because the additional dropping force from gravity will cause further spreading not associated with the intrinsic surface energy of the substrate.  Also, the angled hypodermic tip tends to deform the highly viscous gallium drop resulting in non-uniform hemispheric drop shapes, as shown in Figure \ref{fig:deformed-ga}.
\begin{figure}
	\centering
	\begin{subfigure}[c]{0.45\textwidth}
		\includegraphics[width=\linewidth]{rad-temp-box}
		\subcaption{~}
		\label{fig:rad-temp-box}		
	\end{subfigure}
	\begin{subfigure}[c]{0.45\textwidth} 
		\includegraphics[width=\linewidth]{deformed-ga}
		\subcaption{~}
		\label{fig:deformed-ga}		
	\end{subfigure}
	\caption{(a) The first design of our contact angle goniometer.  The acrylic container houses the argon environment and sample.  This design was modified with a more stable glass enclosure. (b) A highly deformed gallium drop next to the thermocouple on a ceramic YAG test sample at 45.5\degree C.}
	\label{fig:prelim-design}
\end{figure}

For this experiment to succeed, a number of challenges were overcome. The simplest task involved polishing Galfenol samples using incrementally higher grit SiC paper and subsequent 0.1 $\mu$m colloidal silica particles to decrease the roughness to below 40 nm, as proven using AFM measurements. %TODO: Find AFM roughness picture?
% AFM roughness pic
%\begin{figure}
%	\centering
%		\includegraphics[width=\linewidth]{rad-temp-box}
%	\caption{(a) The first design of our contact angle goniometer.  The acrylic container houses the argon environment and sample.  This design was modified with a more stable glass enclosure. (b) A highly deformed gallium drop next to the thermocouple on a ceramic YAG test sample at 45.5\degree C.}
%	\label{fig:prelim-design}
%\end{figure}

The radiative box that housed this experiment had a high variability in temperature caused by opening and closing the box when interaction with the gallium dispensing syringe was needed. A smaller apparatus with a top-side syringe opening is needed to properly perform gallium drop tests at specific temperatures and prevent interaction with the experiment environment. There must also be bright white backlighting to obtain a high contrast drop profile. In this radiative box configuration, the high-reflecting liquid metal surface prevents a high contrast drop profile image, as seen in Figure \ref{fig:deformed-ga}. Proper contact angle measurements also require the gallium droplets to carefully wet the surface while forming an axisymmetric and spherical-like shape on the solid surface. A height adjustment system must be used to move the gallium pendant drop close enough to the sample surface for solid adhesive forces to overcome the adhesion to the needle. Lastly, the argon gas environment could be more well contained, instead of just filling up the glass sample enclosure from the bottom and spilling out the top. This occurs because argon has a higher density than the surrounding air environment. 

\subsubsection{Version 2}

Most of the issues associated with Version 1 of are addressed in Version 2 of the gallium drop experiment. The new experimental apparatus can be seen in Figure \ref{fig:enviro_chamber}. The main structure is made of aluminum with two round glass windows on the front and back. The aluminum is meant to conductively transfer heat to the substrate by means of a heating cable wrapped around the outside of the structure. The high thermal conductivity of aluminum allows for a quick transfer of heat, thus an increased control of sample temperature using the heating tape. The time percentage dial controller attached to the heating tape is calibrated with the sample temperature using a thermocouple placed on the sample surface. Sample temperature can be consistently controlled with $\pm$0.5\degree C accuracy. Backlighting greatly improved the drop profile contrast by having only one white light source coming from one side of the droplet, as seen in Figure \ref{fig:deformed_ga}. The Ar environment is also far more contained and controlled. The silicone sealant creates a nearly air-tight system where the argon will displace all gas contaminants that could oxidize the sample or gallium droplet. These precautions are suitable for any metals samples we test because each sample is polished according to the procedure described above and then cleaned with acetone to remove any oxides. Once the sample is inserted and sealed in the environmental chamber, the Ar prevents oxidation throughout the experiment. A positive partial pressure is achieved in the chamber with an in- and out-valve. %TODO find more sufficient answer to how this removes oxides. 
\begin{figure}
	\centering
	\includegraphics[width=0.7\linewidth]{enviro_chamber}
	\caption{The second version of our gallium contact angle goniometer. The aluminum enclosure conductively transfers heat, the gas lines flow Ar gas into the chamber, top-mounted thermocouples monitor the gas and sample temperature, and the glass windows allow for backlighting of the drop profile along with high resolution image capture using a DSLR camera.}
	\label{fig:enviro_chamber}
\end{figure}
While extensive steps have been taken to inhibit oxidation of our metal samples, preventing oxidation on the surface of liquid gallium was a greater challenge. Pure gallium and Gallium-based alloy surfaces instantly oxidize in ambient environments, turning into a thin layer of gallium oxide (Ga$_{2}$O$_{3}$ and Ga$_{2}$O).\cite{Regan1995,Regan1997,Scharmann2004} This oxide layer is solid and remains elastic until it experiences a yield stress. Therefore, that an oxidized gallium droplet does not behave as a simple liquid, but as a viscoelastic material. In addition, the oxide layer of gallium is known to adhere to almost any solid surface, causing a severe stiction problem that interferes with interfacial energy measurements.\cite{Scharmann2004} Figure \ref{fig:deformed_ga} shows our direct observation of this phenomenon with teardrop shaped droplets formed by adhering to the iron surface while still being pulled upwards by the deposition needles. The general shape of these drops were unchanged for many hours even at temperatures approaching $\sim$100\degree C, thus exhibiting the stability of viscoelastic properties caused by the solid oxide layer. Removing the oxide layer from liquid gallium will return normal liquid properties to gallium and allow the use of axisymmetric drop analysis calculations: Young-Laplace equation, 
%todo: [NEED MORE TECHNIQUES FROM KRUSS GONIOMETER]
Oxide removal permits liquid gallium to directly interact with metal surfaces instead of gallium oxide; the derived terms for \gamSL and \gamLV in Equation \ref{youngs-eqn-ga} become more robust. 

A number of techniques have been developed to remove and recover the oxide layer on liquid gallium,\cite{Kim2013,Doudrick2014,Khan2014} and the most useful strategy for this experiment is a chemical vapor etch. 

Pure liquid gallium obtains viscoelastic properties when trace amounts of oxygen are present via formation of oxide shell. 



\begin{figure}
	\centering
	\begin{subfigure}[c]{0.35\textwidth}
		\includegraphics[width=\linewidth]{ga_41c}
		\subcaption{~}
		\label{fig:ga_41c}		
	\end{subfigure}
	\begin{subfigure}[c]{0.37\textwidth} 
		\includegraphics[width=\linewidth]{ga_100c}
		\subcaption{~}
		\label{fig:ga_100c}		
	\end{subfigure}
	\caption{Pure liquid gallium obtains viscoelastic properties when trace amounts of oxygen are present via formation of oxide shell. Non-axisymmetric Ga drops form on this iron substrate}
	\label{fig:deformed_ga}
\end{figure}

\begin{outline}[enumerate]
	\1 Radiative box [X]
		\2 Plexiglass chamber [X]
		\2 Glass chamber [X]
		\2 Low control of temperature and image clarity [X]
			\3 Need temperature control and backlighting, while improving gas environment. [X]
	\1 Aluminum conductive environmental chamber [~]
		\2 Deformed droplets persist and contact angle cannot be properly measured [~]
		\2 HCl etching to achieve axisymmetric gallium drops [~]
		\2 Surface energy is be measured [~]
		
	\1 MRS Fall Meeting [~]
		\2 Learning from wetting dynamics community [~]
			\3 Static contact angles are not well accepted due to inconsistency  [~]
			\3 make connections to wetting dynamics research group in UMD Mechanical Engineering [~] 
			\3 Learn that gallium drop model is incorrect on a thermodynamic equilibrium level [~]
				\4 Using one thermodynamic equilibrium equation and then plugging values into another thermodynamic equilibrium equation [~]
		
\end{outline}
