This section will describe the 


\subsection{Liquid Gallium drop method}

\textbf{2.1 Surface energy measurement}
There are currently no experimental methods available to measure surface energies associated with different crystallographic orientations of metals. High temperature methods involving cylindrical metal samples and destructive methods involving drop shape samples of a targeted liquid metal near the melting point are available for measuring surface energy of amorphous and isotropic solid metals.\cite{Egry2010,Aqra2011,Cao2011} For our research, however, we require knowledge of orientation-, morphology- and composition-dependent surface energies for validation of DFT results and for use in abnormal grain growth simulations. For glass and polymeric solids with relatively low surface energies, the water-drop method has been shown to be effective for the determination of surface energy at room temperature.\cite{Ahadian2010,Kwok2000,Tavana2005} We attempted to use this approach for contact angle measurements on large 17.9$\%$Ga Galfenol single crystal samples with \hkl(001), \hkl(011) and \hkl(111) grains with facilities at NIST. 


\textcolor{red}{\textbf{INCORRECT STATEMENT}: However, non-physical trends in the results lead to the conclusion that disruption of strong inter-atomic and molecular bonds with high coordination numbers at the free substrate surface (an intrinsic property of a metal substrate[75,76,77]) interfered with the water drop surface tension and resultant water drop-substrate contact angles.}
%TODO: Finish writing about why the water drop method will not work for metals.

Water will completely wet the high surface energy of metals.
%TODO: Explain why water completely wets a high energy surface

We proposed using a liquid gallium droplet to measure the orientation-dependent surface energy of high-surface energy metals, like Galfenol and Alfenol. This project has been funded by the \NSF. 


\subsection{Derivation of Method}

To properly measure surface energy of specific grains of a metal, we proposed technique that employs measurement of contact angles and size of a drop of a liquid metal, gallium, which has a melting point of ~30oC, resting on a metal surface as the metal is heated. Changes in surface tension will be introduced by controlled thermal expansion of the substrate. Measurements will be performed at temperatures ranging $\sim$30-100\degree C to give mobility to the molten gallium drop and ensure that surface heterogeneities do not obscure the drop-substrate contact angle. We will employ a video system to observe the gallium melting process (formation of a liquid drop from a solid) and to precisely quantify dimension changes in substrate and liquid metal drop during thermal expansion. The surface energy of the free surface of the substrate as a function of temperature, \gamSV(T), can be related to thermal-expansion-induced changes in the liquid gallium drop contact angle $\theta$, the radius $r$ of the liquid gallium drop in contact with the metal surface, and the height $h$ of the hemisphere of liquid gallium. 
\begin{figure}
	\centering
	\begin{subfigure}[b]{0.5\textwidth}
		\includegraphics[width=\textwidth]{youngs-ga}
	%	\caption{Interfacial tensions on Ga drop on solid surface}
		\label{fig:youngs-ga}
	\end{subfigure}
	%add desired spacing between images, e. g. ~, \quad, \qquad, \hfill etc. 
	%(or a blank line to force the subfigure onto a new line)
	\begin{subfigure}[b]{0.5\textwidth}
		\includegraphics[width=\textwidth]{thermal-expand-drop}
	%	\caption{Thermal expansion of drop}
		\label{fig:thermal-expand-drop}
	\end{subfigure}
	\caption{Schematics indicating notation used and contact angles $\theta_{i}$, radii of drop-solid contact area $r_{i}$, and radii of curvature for a spherical drop $R_{i}$, for two temperatures as thermal expansion induced tension $P$ strains the substrate.}
	\label{fig:therm-exp-ga}
\end{figure}
These can be modeled as being due to a tensile load +$P(T)$ caused by substrate thermal expansion (Figure \ref{fig:therm-exp-ga}), which effectively appears as a uniform radial expansion of the planar solid surface as temperature increases. By letting the system equilibrate at different temperatures before measuring liquid drop geometry, terms due to variation in thermal energy and variation of the total Gibbs free energy vanish. Then, the interface energy between the substrate solid and the liquid gallium drop, \gamSL, is given as \gamSL$=P(T)/2\pi r$. The uniform tension introduced by thermal expansion of substrate can be expressed as:
\begin{equation}\label{uniform-tension}
	P(T) = E_{sub}\alpha_{sub}(T-T_{mp})
\end{equation}
where $T_{mp}$ is the melting temperature of gallium, $E$  is the substrate Young’s modulus, and $\alpha$ is the linear thermal expansion coefficient. We can use a linear function to write the gallium-air interfacial tension a function of temperature, \gamLV $= a-b(T-T_{mp})$, where a and b are positive constants.\cite{Hardy1985,Alchagirov2005} Putting the terms for \gamSL and \gamSV into Young’s equation \ref{youngs-eqn} we get\cite{Rudawska2009,Tadmor2004}:
\begin{equation*}%\label{youngs-eqn-ga1}
	\gamma_{SV} =  \frac{E_{sub}\alpha_{sub}(T-T_{mp})}{2\pi r} + \left[a-b(T-T_{mp})\right]\cos\theta
\end{equation*}

We derived the relationship between the variable radius $r(T)$ associated with the area of the circular region of solid-liquid contact $A_{SL}(T)$, the contact angle $\theta(T)$ and the volume $V$ of the spherical cap formed by the drop. To determine the radius $r(T)$, we use geometric relationships based on the radius of curvature $R(T)$ of a sphere  that can be mapped onto the hemispherical liquid cap. The drop is modeled as being part of a sphere whose radius $R(T)$ we want to find. It is assumed that he volume $V$ of the drop is the volume of the spherical cap, and remains constant. The radius $R(T)$ can then be expressed in terms of the volume and the angle:
\begin{equation*}\label{drop-geom}
	R(T) = V^{1/3} \left[\frac{\pi}{3} \left(2-3\cos\theta(T)+\cos^{3}\theta(T)\right)\right]^{-1/3}
\end{equation*}
Using the relation $r(T)=R(T)\sin\theta(T)$ (i.e. $\theta=$0\degree corresponds to complete wetting of the surface and at $
theta=90$\degree $r=R$) we can obtain a formula for surface energy as a function of temperature $T$ and contact angle $\theta(T)$ (shown as $\theta_{T}$):
\begin{equation}\label{youngs-eqn-ga}
	\gamma_{SV} =  \frac{E_{sub}\alpha_{sub}(T-T_{mp})}{2\pi}\left[\frac{\pi\left(2-3\cos\theta_{T}+\cos^{3}\theta_{T}\right)}{3V\sin^{3}\theta_{T}} \right]^{1/3} + \left[a-b(T-T_{mp})\right]\cos\theta_{T}
\end{equation}
This will allow us to obtain the surface energy \gamSV of specific grains as a function of temperature by measuring $T$ and $\theta$ at thermal equilibrium.

One of the challenges to be addressed in developing this new surface energy measurement capability is the need to ensure that the surface of the metal substrate is not shielded by surface oxides. We propose to use the following strategies for removing of oxide layer and preventing of natural oxidation to increase the accuracy of measurement. The following chemical, electrical and/or mechanical methods will be employed independently and in combination:
\begin{outline}
	\1 A flux used in high-temperature metal joining processes plays roles of dissolving of the oxides on the metal surface and preventing of re-oxidation as a chemical agent.
		
	\1 Colloidal silica polishing with nano-sized particles, such as is used for precise surface observations, like EBSD scans, which require clean surfaces to accurately detect patterns.
	\1 Electro-polishing is effective for passivation of clean surfaces after chemical and mechanical polishing for removal of surface oxides. 
	\1 \textcolor{red}{All of these techniques will not prevent oxides from forming for an extended amount of time. Therefore, the cleaning would have to be followed by isolation in high vacuum, an inert environment, or another liquid environment that prevents oxidation.} 
\end{outline}
%Results using the proposed approach were to be validated through comparison of results for oriented single crystals with published theoretical values for pure metal elements of a specific crystallography (e.g. Ni, Cu, Fe)  and comparison with experimental data from amorphous metals (e.g. Vitreloy or liquid steel) with destructive high temperature methods for measuring surface energy. 

Once experimental results are validated with published values, we will test and verify DFT predictions of our iron-based alloys. Finally, this method has the potential to become a valuable tool for obtaining empirical information on the surface energies associated with AGG and highly textured conditions on sulfur-contaminated Galfenol textured sheet after different anneal temperatures, durations and atmospheres. These results will supplement insights used for AGG model enhancement. This task and the resultant capability should have broad applicability beyond the needs of this proposal for extension to polycrystalline textured metals and thin films.







\subsection{Experimental Versions}
%todo: Go through each version, and mark what was improved upon from experimental progression. 

\subsubsection{Thermal Grooving}



\begin{figure}[h!]
	\centering
	\begin{subfigure}[c]{0.45\textwidth}
		\includegraphics[width=\linewidth]{afm-groove-fega}
		\subcaption{~}
		\label{fig:afm-groove-fega}		
	\end{subfigure}
	\begin{subfigure}[c]{0.45\textwidth} 
		\includegraphics[width=\linewidth]{fega-groove-profile}
		\subcaption{~}
		\label{fig:fega-groove-profile}		
	\end{subfigure}
	\caption{(a) 3D rendering of (110)/(111) grain boundary on surface of  for (Fe-19\%Ga)+1.0\%NbC sample where the depth of groove is ~8 nm. (b) A quadratic fit of a (110)/(111) grain boundary profile for a (Fe-19\%Ga)+1.0\%NbC sample.}
	\label{fig:thermal-groove}
\end{figure}

\begin{table}[h!]
	\centering
	\caption{Calculated dihedral angles and relative energies from our most symmetric grain boundary groove.}
	\begin{tabular} { |p{1cm}||c|c|c|c|  } 
		\hline
		\multicolumn{5}{|c|}{fe-ga-s12-006 profile analysis - GB (110)/(111)}\\
		\hline
		~	&\multicolumn{2}{|c|}{Bruker Software}		&\multicolumn{2}{|c|}{Quadratic Fit}	\\
		\hline
		Profile	&Dihedral Angle (\degree)	&Relative GB Energy	&Dihedral Angle (\degree)	&Relative GB Energy \\ 
		\hline
		1		&156.957	&0.399471	&151.253	&0.496475	\\
		\hline
		2		&156.244	&0.411657	&157.093	&0.397144	\\
		\hline
		3		&154.402	&0.443063	&155.554	&0.423432	\\
		\hline
		4		&157.221	&0.394955	&152.785	&0.470541	\\
		\hline
		5		&154.732	&0.437445	&154.966	&0.433458	\\
		\hline
		6		&158.386	&0.375003	&154.482	&0.441706	\\
		\hline
		\textbf{Avg}	&156.324$\pm$1.52919	&0.410266$\pm$0.0261178	&154.356$\pm$2.06926	&0.443793$\pm$0.0351932\\
		\hline
	\end{tabular}
	\label{groove-analysis}
\end{table}


The thermal grooving technique was examined to provide a comparison point to our proposed gallium contact angle method.  These thermal grooves appear at grain boundaries, but the most information can be drawn from grain boundaries formed by two different crystal orientations. Using electron backscatter diffraction, EBSD, we identified grain orientations on samples of galfenol and alfenol to identify the grain boundaries where \hkl(100), \hkl(110), and \hkl(111) orientations met.  By measuring the dihedral angle formed at these grain boundaries after polishing and annealing, we calculated the ratio of grain boundary energy and surface energy according to the following equation: 
\begin{equation}
\frac{\gamma_{GB}}{\gamma_{S}} = 2\cos\left(\frac{\Psi_{S}}{2}\right) 
\end{equation}
where $\gamma_{GB}$ is the grain boundary energy, $\gamma_{S}  $ is the surface energy, and $\Psi_{S} $ is the dihedral angle, as described in Rohrer et al.\cite{Rohrer2010a} The most symmetric thermal groove came from a \hkl(110)/\hkl(111) grain boundary on a rolled and annealed (Fe-19\%Ga)+1.0\%NbC sample made by Suok-Min Na, as seen in the Figure \ref{fig:thermal-groove}.  The grain boundary profiles were measured using atomic force microscopy (AFM), and the dihedral angles were extrapolated from the profiles using both AFM software by Bruker and, the more accurate, quadratic fit.  The results are shown in Table \ref{groove-analysis}.  This groove in particular had a depth of $\sim$8 nm which is significantly smaller in depth compared to grooves of other metal alloys in literature.  Also, many of the grain boundary grooves were not suitable for measuring based on their lack of symmetry at the grain boundary interface. It is worthy to note that this technique is very time consuming and slightly destructive to the surface. To properly analyze the thermal grooving technique, an extensive grain boundary study of annealing temperatures and times on Alfenol and Galfenol would have to be carried out. While this may be an interesting avenue of research in the future, the resultant calculations of relative grain energies does not benefit our ultimate goal of achieving a comprehensive AGG model for Galfenol and Alfenol. Realization of our goal lies in the measurement of orientation-dependent surface energy using contact angle measurements. 
	


\subsubsection{Contact Angle Goniometer}
Preliminary designs of our contact angle goniometer implement a radiative temperature control box which encloses an argon gas filled container where the sample resides, as seen in Figure \ref{fig:rad-temp-box}.  The argon filled container was initially made of a clear acrylic plastic, but prolonged exposure to temperatures above 80\degree C caused thermal deformation of the plastic making longer experiments impossible to perform without environment contamination.  A clear pyrex container replaced the acrylic box to fix this issue.  Application of the liquid gallium drop to our surfaces was done via a mounted plastic syringe with commercially available disposable stainless steel hypodermic needles.  Observations of this application show that the hypodermic needles present multiple problems to the sessile drop method.  Liquid gallium tends to adhere strongly to the stainless steel needle tips which makes wetting to the sample very difficult.  Tapping the syringe will remove the drop from the needle and allow gallium to wet the surface, but this is not ideal because the additional dropping force from gravity will cause additional spreading not associated with the intrinsic surface energy of the substrate.  Also, the angled hypodermic tip tends to deform the highly viscous gallium drop resulting in non-uniform hemispheric drop shapes, as shown in Figure \ref{fig:deformed-ga}.
\begin{figure}
	\centering
	\begin{subfigure}[c]{0.45\textwidth}
		\includegraphics[width=\linewidth]{rad-temp-box}
		\subcaption{~}
		\label{fig:rad-temp-box}		
	\end{subfigure}
	\begin{subfigure}[c]{0.45\textwidth} 
		\includegraphics[width=\linewidth]{deformed-ga}
		\subcaption{~}
		\label{fig:deformed-ga}		
	\end{subfigure}
	\caption{(a) The first design of our contact angle goniometer.  The acrylic container houses the argon environment and sample.  This design was modified with a more stable glass enclosure. (b) A highly deformed gallium drop next to the thermocouple on a ceramic YAG test sample at 45.5\degree C.}
	\label{fig:prelim-design}
\end{figure}

For this experiment to succeed, a number of challenges were overcome. The simplest task involved polishing Galfenol samples using incrementally higher grit SiC paper and subsequent 0.1 $\mu$m colloidal silica particles to decrease the roughness to below 100 nm, as proven using AFM measurements. [FIND AFM PICTURE OF ROUGHNESS]

Gallium droplets must carefully wet the surface while forming an axisymmetric and spherical-like droplet on the solid surface. 

\begin{outline}[enumerate]
	\1 Radiative box [~]
		\2 Plexiglass chamber [X]
		\2 Glass chamber [X]
		\2 Low control of temperature and image clarity [~]
			\3 Need temperature control and backlighting, while improving gas environment. [~]
	\1 Aluminum conductive environmental chamber [~]
		\2 Deformed droplets persist and contact angle cannot be properly measured [~]
		\2 HCl etching to achieve axisymmetric gallium drops [~]
		\2 Surface energy can be measured [~]
	\1 MRS Fall Meeting [~]
		\2 Learning from wetting dynamics community [~]
			\3 Static contact angles are not well accepted due to inconsistency  [~]
			\3 make connections to wetting dynamics research group in UMD Mechanical Engineering [~] 
			\3 Learn that gallium drop model is incorrect on a thermodynamic equilibrium level [~]
				\4 Using one thermodynamic equilibrium equation and then plugging values into another thermodynamic equilibrium equation [~]
		
\end{outline}


\subsection{Results and Presentations}

%TODO: Explain why this method is not possible from a fundamental thermodynamic position (ie using one thermodynamic equilibirum and plugging it in to another equation at thermodynamic equilibrium). Should be able to go from one thermodynamic equilibrium equation to another, but that is not the case here. The method could still be feasible if more terms (\gamSL between liquid-Ga and solid) were known
